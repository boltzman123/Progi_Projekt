\chapter{Opis projektnog zadatka}
		
		%\textbf{\textit{dio 1. revizije}}\\
		
		\textit{Na osnovi projektnog zadatka detaljno opisati korisničke zahtjeve. Što jasnije opisati cilj projektnog zadatka, razraditi problematiku zadatka, dodati nove aspekte problema i potencijalnih rješenja. Očekuje se minimalno 3, a poželjno 4-5 stranica opisa.	Teme koje treba dodatno razraditi u ovom poglavlju su:}
		\begin{packed_item}
			\item \textit{Cilj projekta „Djeca za djecu“ je razviti web aplikaciju koja će omogućiti roditeljima, i onim budućim, da lakše doniraju i pronalaze donacije za svoju djecu. 
U doba kada su već osnovne namirnice preskupe i globalno zagađenje raste, „recikliranje“ predmeta postaje još važnije. Zašto bacati dobro očuvane stvari koje još mogu poslužiti nekome i spasiti njegov džep, a nemamo ih gdje čuvati dok ne zatrebaju kolegi/prijateljima/rodbini…  
Ova aplikacija brzo i efikasno spaja donatora i primatelja donacije bez da korisnici moraju pretraživati bespuća interneta. Oglasi u aplikaciji prilagođavaju se profilu korisnika. Nakon što korisnik jednom upiše svoje podatke i podatke o svojoj djeci korisniku se automatski prikazuju oglasi od interesa. Dodatno se mogu namještati filteri kategorija kod pregleda oglasa. 
Doniranje, također, nikada nije bilo lakše. Nakon što korisnik objavi oglas za predmet više se ne mora brinuti o njemu dok aplikacija sama ne pronađe savršenog primatelja. Tek kada predmet pronađe svog potencijalnog novog vlasnika zamišljeno je da se primatelj kontaktira donatora izvan aplikacije te se dogovori za primopredaju. Nakon izvršene donacije donator treba samo potvrditi da je donirao korisniku upisivanjem korisničkog imena primatelja. 
Sigurnost i kredibilitet aplikacije i oglasa je važan za iskustvo korisnika. Stoga je aplikacija zatvorena na registrirane korisnike što omogućava praćenje aktivnosti unutar aplikacije. Provode se provjere na razini korisnika i oglasa. Korisnici (ADMIN van aplikacije) se provjeravaju na temelju prijašnjih donacija ili postojećih zapisa o njima kako bi eliminirali moguće prevare ostalih korisnika aplikacije. Oglasi, pak, moraju biti opisani u skladu sa standardom aplikacije i predmeti ne stariji od preporučene starosti za određen predmet. 
}
			\item \textit{postojeća slična rješenja (istražiti i ukratko opisati razlike u odnosu na zadani zadatak). Dodajte slike koja predočavaju slična rješenja.}
			\item \textit{skup korisnika koji bi mogao biti zainteresiran za ostvareno rješenje.}
			\item \textit{mogućnost prilagodbe rješenja }
			\item \textit{opseg projektnog zadatka}
			\item \textit{moguće nadogradnje projektnog zadatka}
		\end{packed_item}
		
		\textit{Za pomoć pogledati reference navedene u poglavlju „Popis literature“, a po potrebi konzultirati sadržaj na internetu koji nudi dobre smjernice u tom pogledu.}
		\eject
		