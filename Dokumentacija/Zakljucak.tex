\chapter{Zaključak i budući rad}
		
		%\textbf{\textit{dio 2. revizije}}\\

		Zadatak naše grupe bio je razvoj web aplikacije za razmjenu dječjih predmeta i stvari. Cilj aplikacije je bio omogućiti lakše doniranje dječjih predmeta, raznih kategorija od igračaka do kolica i robe. Nakon 11 tjedana intenzivnog rada na projektu uspjeli smo ostvariti sve zahtijevane funkcionalnosti i poneku više. Projekt je bio organiziran u dvije velike faze, koje su, jasno, bile podijeljene na manje faze radi boljeg održavanja tempa i jasnijih ciljeva. \\
  \\
Prva faza započela je formiranjem tima od sedmero studenata za koji će u narednih 11 tjedana razvijati aplikaciju, učiti odabrane alate, pisati dokumentaciju i naučiti održavati sastanke. Dodjela projektnog zadatka i upoznavanje asistentice uslijedila je ubrzo nakon. Bilo je važno dobro analizirati zadatak i izvući sve zahtjeve iz opisa zadatka. U prvoj je fazi bio veći naglasak na izradi dokumentacije koja je predstavljala ključ priprema za implementiranje aplikacije. Iz pažljivo prepoznatih funkcionalnih zahtjeva napravljeni su dijagrami obrazaca uporabe i sekvencijski dijagrami koji su bili referentni "dokument" za implementaciju. Promišljenim modelom baze podataka također je olakšana i ubrzana implementacija baze. Jednom kada je promišljanje o arhitekturi i izgledu aplikacije bilo gotovo moglo se krenuti u drugu fazu u kojoj nije bilo vremena za gubiti i nismo ga izgubili jer smo imali dobru pripremu. \\
\\
Druga faza je trajala kraće, ali je bila nešto intenzivnija od prve što se tiče samostalnog učenja tehnologija i kodiranja. Dokumentacija se razvijala paralelno s aplikacijom i u nju je trebalo dodati preostale UML dijagrame vezane za arhitekturu sustava. Sastanci tima postali su češći kako bi bili sigurni da se sve odvija po planu. Izrađen je i dio dokumentacije koji će omogućiti budućim korisnicima lakše snalaženje i korištenje aplikacije kao i eventualnu nadogradnju sustava. \\
\\
Zadovoljni smo završnom verzijom aplikacije i suradnjom koja je trajala ovih 11 tjedana. Imamo ideje za budući rad u vidu vizualnih poboljšanja, ali i tehničkih/ funkcionalnih poboljšanja. Glavna ideja bila bi dodati kartu u oglas kako bi se odmah prikazivala adresa na mapi i mogla otvoriti karta u Google Maps aplikaciji. Također bi time olakšali posao adminu koji ne bi morao provjeravati postoji li adresa jer se u kreiranju oglasa ne bi mogla dodati adresa koju Google Maps ne prepoznaje. Druga ideja bila bi sustav za prijavljianje korisnika koji bi adminu olakšao dodijeljivanje prava na kreiranje oglasa kao i povećao zadovoljstvo korisnika. Kako bi aplikacija bila kompletna ideja je još dodati "chat" unutar aplikacije između osobe koja donira i osobe koja želi preuzeti donaciju. Imamo još puno sitnih ideja, ali ove su glavne pa smo njih odlučili spomenuti.  \\
\\
Kroz ovaj projekt smo naučili da je dobar tim i komunikacija u timu izrazito važna. Ispravna (kako se pokazalo kasnije) podjela u podtimove i procjena količine posla učinile su da tim funkcionira jako dobro. Komunicirali smo često putem WhatsApp grupe, a sastanke smo, osim uživo, imali preko Discorda. Motivacija tima je bila visoka i nije bilo problema između članova što je značajno za dobar napredak projekta. \\
\\
Rad na ovakvom projektu nam je pokazao koliko je važno znati organizirati vrijeme i izvršavati svoje zadatke na vrijeme, ali i koliko znači imati prave ljude u timu i dobro vodstvo. Kontrole s mentoricom su nam puno značile i njezini komentari koji su nas usmjeravali i motivirali da budemo što bolji kao tim i razvijemo što kvalitetniju aplikaciju. \\
  
     

		% \textit{U ovom poglavlju potrebno je napisati osvrt na vrijeme izrade projektnog zadatka, koji su tehnički izazovi prepoznati, jesu li riješeni ili kako bi mogli biti riješeni, koja su znanja stečena pri izradi projekta, koja bi znanja bila posebno potrebna za brže i kvalitetnije ostvarenje projekta i koje bi bile perspektive za nastavak rada u projektnoj grupi.}
		
		 %\textit{Potrebno je točno popisati funkcionalnosti koje nisu implementirane u ostvarenoj aplikaciji.}
		
		\eject 